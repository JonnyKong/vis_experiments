\documentclass[11pt,a4paper]{article}
\usepackage{amsmath}
\usepackage{url}
\usepackage{graphicx} % Required for including pictures
\usepackage{microtype} % Improves typography
\usepackage{gfsdidot} % Use the GFS Didot font: http://www.tug.dk/FontCatalogue/gf⇉
\usepackage[T1]{fontenc} % Required for accented characters
\usepackage{fullpage}
\usepackage{hyperref}
\usepackage{enumitem}
\usepackage{listings}
\usepackage{cleveref}
\usepackage{booktabs} % For formal tables
\usepackage{xspace}
\usepackage{xcolor}
\usepackage{microtype}
\usepackage{subcaption}
\usepackage{balance}
%\usepackage{mathtools}
\usepackage{scalerel,amssymb}

\newcommand{\ititle}[1]{\vspace{2pt}\noindent\emph{#1:\xspace}}
\newcommand{\stitle}[1]{\smallskip\noindent\textbf{#1:\xspace}}
\newcommand{\sstitle}[1]{\noindent\textbf{#1:\xspace}}
\DeclareMathOperator*{\argmax}{arg\,max}
\DeclareMathOperator*{\argmin}{arg\,min}



\newtheorem{thm}{Theorem}
\newtheorem{example}[thm]{Example}
\newtheorem{definition}[thm]{Definition}
\newtheorem{problem}[thm]{Problem}


\lstset{
  basicstyle=\ttfamily,
  columns=fullflexible,
}

\begin{document}

\section*{Visualization Priority}

\begin{problem}[Visualization Priority]
  A syntax for developers to specify visualization constrains and priorities, in order to come up with placement policies.
\end{problem}

Suppose we have:\\

\hspace{10mm} Visualizations $V = \{V_1, V_2, ..., V_n\}$\\

\hspace{10mm} Interactions $I = \{I_1, I_2, ..., I_n\}$\\

where $V$ and $I$ have many-to-many relations. For example, we might have a view that can be updated by multiple different interactions, and we can have an interaction that changes multiple views simultaneously. This means prioritization should be expressed at the granularity of <$V_m$, $I_n$> pairs.\\

Specifically, there could be hard and soft constraints. Developers specify hard constraints like ``Interaction $I_1$ should update $V_2$ in 100ms for 95\% of the time”, and soft constraints like “Interaction $I_1$ updating $V_1$ should be faster than $I_2$ updating $V_2$”. \\

For hard constraints, the goal would be to find the lowest priority that meets each of the constrains, if there exists such a placement. \\

If developers don't have specific latency bounds in mind, they can specify soft constraints. A naive approach would be to let the developer to specify the weights of selected <$V_m$, $I_n$> pairs, and the system calculates a placement policy that minimizes the weighted average latency based on remaining resources. The execution time should also be considered. \\

An example spec would be:\\

Hard:

\textit{($I_1$, $V_1$) within 100ms with 100\% confidence}

\textit{($I_2$, *) within 200ms with 95\% confidence}\\

Soft:

\textit{($I_1$, $V_1$) weight 2}

\textit{($I_1$, $V_2$) weight 3}

\textit{(*, $V_1$) weight 5}\\

 For example, one may use "\textit{(*, $V_1$) weight 1, (*, $V_2$) weight 1}" to suggest that ``most interactions that updates $V_1$ and $V_2$ should be prioritized".

\end{document}